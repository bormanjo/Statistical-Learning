\documentclass[]{article}
\usepackage{lmodern}
\usepackage{amssymb,amsmath}
\usepackage{ifxetex,ifluatex}
\usepackage{fixltx2e} % provides \textsubscript
\ifnum 0\ifxetex 1\fi\ifluatex 1\fi=0 % if pdftex
  \usepackage[T1]{fontenc}
  \usepackage[utf8]{inputenc}
\else % if luatex or xelatex
  \ifxetex
    \usepackage{mathspec}
  \else
    \usepackage{fontspec}
  \fi
  \defaultfontfeatures{Ligatures=TeX,Scale=MatchLowercase}
\fi
% use upquote if available, for straight quotes in verbatim environments
\IfFileExists{upquote.sty}{\usepackage{upquote}}{}
% use microtype if available
\IfFileExists{microtype.sty}{%
\usepackage{microtype}
\UseMicrotypeSet[protrusion]{basicmath} % disable protrusion for tt fonts
}{}
\usepackage[margin=1in]{geometry}
\usepackage{hyperref}
\hypersetup{unicode=true,
            pdfborder={0 0 0},
            breaklinks=true}
\urlstyle{same}  % don't use monospace font for urls
\usepackage{color}
\usepackage{fancyvrb}
\newcommand{\VerbBar}{|}
\newcommand{\VERB}{\Verb[commandchars=\\\{\}]}
\DefineVerbatimEnvironment{Highlighting}{Verbatim}{commandchars=\\\{\}}
% Add ',fontsize=\small' for more characters per line
\usepackage{framed}
\definecolor{shadecolor}{RGB}{248,248,248}
\newenvironment{Shaded}{\begin{snugshade}}{\end{snugshade}}
\newcommand{\KeywordTok}[1]{\textcolor[rgb]{0.13,0.29,0.53}{\textbf{#1}}}
\newcommand{\DataTypeTok}[1]{\textcolor[rgb]{0.13,0.29,0.53}{#1}}
\newcommand{\DecValTok}[1]{\textcolor[rgb]{0.00,0.00,0.81}{#1}}
\newcommand{\BaseNTok}[1]{\textcolor[rgb]{0.00,0.00,0.81}{#1}}
\newcommand{\FloatTok}[1]{\textcolor[rgb]{0.00,0.00,0.81}{#1}}
\newcommand{\ConstantTok}[1]{\textcolor[rgb]{0.00,0.00,0.00}{#1}}
\newcommand{\CharTok}[1]{\textcolor[rgb]{0.31,0.60,0.02}{#1}}
\newcommand{\SpecialCharTok}[1]{\textcolor[rgb]{0.00,0.00,0.00}{#1}}
\newcommand{\StringTok}[1]{\textcolor[rgb]{0.31,0.60,0.02}{#1}}
\newcommand{\VerbatimStringTok}[1]{\textcolor[rgb]{0.31,0.60,0.02}{#1}}
\newcommand{\SpecialStringTok}[1]{\textcolor[rgb]{0.31,0.60,0.02}{#1}}
\newcommand{\ImportTok}[1]{#1}
\newcommand{\CommentTok}[1]{\textcolor[rgb]{0.56,0.35,0.01}{\textit{#1}}}
\newcommand{\DocumentationTok}[1]{\textcolor[rgb]{0.56,0.35,0.01}{\textbf{\textit{#1}}}}
\newcommand{\AnnotationTok}[1]{\textcolor[rgb]{0.56,0.35,0.01}{\textbf{\textit{#1}}}}
\newcommand{\CommentVarTok}[1]{\textcolor[rgb]{0.56,0.35,0.01}{\textbf{\textit{#1}}}}
\newcommand{\OtherTok}[1]{\textcolor[rgb]{0.56,0.35,0.01}{#1}}
\newcommand{\FunctionTok}[1]{\textcolor[rgb]{0.00,0.00,0.00}{#1}}
\newcommand{\VariableTok}[1]{\textcolor[rgb]{0.00,0.00,0.00}{#1}}
\newcommand{\ControlFlowTok}[1]{\textcolor[rgb]{0.13,0.29,0.53}{\textbf{#1}}}
\newcommand{\OperatorTok}[1]{\textcolor[rgb]{0.81,0.36,0.00}{\textbf{#1}}}
\newcommand{\BuiltInTok}[1]{#1}
\newcommand{\ExtensionTok}[1]{#1}
\newcommand{\PreprocessorTok}[1]{\textcolor[rgb]{0.56,0.35,0.01}{\textit{#1}}}
\newcommand{\AttributeTok}[1]{\textcolor[rgb]{0.77,0.63,0.00}{#1}}
\newcommand{\RegionMarkerTok}[1]{#1}
\newcommand{\InformationTok}[1]{\textcolor[rgb]{0.56,0.35,0.01}{\textbf{\textit{#1}}}}
\newcommand{\WarningTok}[1]{\textcolor[rgb]{0.56,0.35,0.01}{\textbf{\textit{#1}}}}
\newcommand{\AlertTok}[1]{\textcolor[rgb]{0.94,0.16,0.16}{#1}}
\newcommand{\ErrorTok}[1]{\textcolor[rgb]{0.64,0.00,0.00}{\textbf{#1}}}
\newcommand{\NormalTok}[1]{#1}
\usepackage{graphicx,grffile}
\makeatletter
\def\maxwidth{\ifdim\Gin@nat@width>\linewidth\linewidth\else\Gin@nat@width\fi}
\def\maxheight{\ifdim\Gin@nat@height>\textheight\textheight\else\Gin@nat@height\fi}
\makeatother
% Scale images if necessary, so that they will not overflow the page
% margins by default, and it is still possible to overwrite the defaults
% using explicit options in \includegraphics[width, height, ...]{}
\setkeys{Gin}{width=\maxwidth,height=\maxheight,keepaspectratio}
\IfFileExists{parskip.sty}{%
\usepackage{parskip}
}{% else
\setlength{\parindent}{0pt}
\setlength{\parskip}{6pt plus 2pt minus 1pt}
}
\setlength{\emergencystretch}{3em}  % prevent overfull lines
\providecommand{\tightlist}{%
  \setlength{\itemsep}{0pt}\setlength{\parskip}{0pt}}
\setcounter{secnumdepth}{0}
% Redefines (sub)paragraphs to behave more like sections
\ifx\paragraph\undefined\else
\let\oldparagraph\paragraph
\renewcommand{\paragraph}[1]{\oldparagraph{#1}\mbox{}}
\fi
\ifx\subparagraph\undefined\else
\let\oldsubparagraph\subparagraph
\renewcommand{\subparagraph}[1]{\oldsubparagraph{#1}\mbox{}}
\fi

%%% Use protect on footnotes to avoid problems with footnotes in titles
\let\rmarkdownfootnote\footnote%
\def\footnote{\protect\rmarkdownfootnote}

%%% Change title format to be more compact
\usepackage{titling}

% Create subtitle command for use in maketitle
\newcommand{\subtitle}[1]{
  \posttitle{
    \begin{center}\large#1\end{center}
    }
}

\setlength{\droptitle}{-2em}

  \title{}
    \pretitle{\vspace{\droptitle}}
  \posttitle{}
    \author{}
    \preauthor{}\postauthor{}
    \date{}
    \predate{}\postdate{}
  

\begin{document}

\section{FE590. Assignment \#1.}\label{fe590.-assignment-1.}

\subsection{2018-09-07}\label{section}

\section{Question 1}\label{question-1}

\subsection{Question 1.1}\label{question-1.1}

\begin{Shaded}
\begin{Highlighting}[]
\NormalTok{CWID =}\StringTok{ }\DecValTok{10402229} \CommentTok{#Place here your Campus wide ID number, this will personalize}
\CommentTok{#your results, but still maintain the reproduceable nature of using seeds.}
\CommentTok{#If you ever need to reset the seed in this assignment, use this as your seed}
\CommentTok{#Papers that use -1 as this CWID variable will earn 0's so make sure you change}
\CommentTok{#this value before you submit your work.}
\NormalTok{personal =}\StringTok{ }\NormalTok{CWID }\OperatorTok\StringTok{ }\DecValTok{10000}
\KeywordTok{set.seed}\NormalTok{(personal)}
\end{Highlighting}
\end{Shaded}

Generate a vector \texttt{x} containing 10,000 realizations of a random
normal variable with mean 2.0 and standard deviation 3.0, and plot a
histogram of \texttt{x} using 100 bins. To get help generating the data,
you can type \texttt{?rnorm} at the R prompt, and to get help with the
histogram function, type \texttt{?hist} at the R prompt.

\subsection{\texorpdfstring{\textcolor{red}{Solution:}}{}}\label{section-1}

\begin{Shaded}
\begin{Highlighting}[]
\NormalTok{x <-}\StringTok{ }\KeywordTok{rnorm}\NormalTok{(}\DataTypeTok{n =} \DecValTok{10000}\NormalTok{, }\DataTypeTok{mean =} \FloatTok{2.0}\NormalTok{, }\DataTypeTok{sd =} \FloatTok{3.0}\NormalTok{)}
\KeywordTok{hist}\NormalTok{(x)}
\end{Highlighting}
\end{Shaded}

\includegraphics{A1_files/figure-latex/unnamed-chunk-2-1.pdf}

\subsection{Question 1.2}\label{question-1.2}

Confirm that the mean and standard deviation are what you expected using
the commands \texttt{mean} and \texttt{sd}.

\subsection{\texorpdfstring{\textcolor{red}{Solution:}}{}}\label{section-2}

\begin{Shaded}
\begin{Highlighting}[]
\KeywordTok{mean}\NormalTok{(x)}
\end{Highlighting}
\end{Shaded}

\begin{verbatim}
## [1] 1.982375
\end{verbatim}

\begin{Shaded}
\begin{Highlighting}[]
\KeywordTok{sd}\NormalTok{(x)}
\end{Highlighting}
\end{Shaded}

\begin{verbatim}
## [1] 3.018047
\end{verbatim}

\subsection{Question 1.3}\label{question-1.3}

Using the \texttt{sample} function, take out 10 random samples of 500
observations each. Calculate the mean of each sample. Then calculate the
mean of the sample means and the standard deviation of the sample means.

\subsection{\texorpdfstring{\textcolor{red}{Solution:}}{}}\label{section-3}

\begin{Shaded}
\begin{Highlighting}[]
\ControlFlowTok{for}\NormalTok{(i }\ControlFlowTok{in} \DecValTok{1}\OperatorTok{:}\DecValTok{10}\NormalTok{)\{}
\NormalTok{  sample_vec <-}\StringTok{ }\KeywordTok{sample}\NormalTok{(}\DataTypeTok{x =}\NormalTok{ x, }\DataTypeTok{size =} \DecValTok{500}\NormalTok{)}
  \KeywordTok{print}\NormalTok{(}\KeywordTok{paste0}\NormalTok{(}\StringTok{"Iteration: "}\NormalTok{, i))}
  \KeywordTok{print}\NormalTok{(}\KeywordTok{paste0}\NormalTok{(}\StringTok{"mean: "}\NormalTok{, }\KeywordTok{mean}\NormalTok{(sample_vec)))}
  \KeywordTok{print}\NormalTok{(}\KeywordTok{paste0}\NormalTok{(}\StringTok{"Std Dev: "}\NormalTok{, }\KeywordTok{sd}\NormalTok{(sample_vec)))}
\NormalTok{\}}
\end{Highlighting}
\end{Shaded}

\begin{verbatim}
## [1] "Iteration: 1"
## [1] "mean: 1.89762886557256"
## [1] "Std Dev: 3.02585952729847"
## [1] "Iteration: 2"
## [1] "mean: 1.93741578107116"
## [1] "Std Dev: 3.10449893128094"
## [1] "Iteration: 3"
## [1] "mean: 1.9120817890946"
## [1] "Std Dev: 2.90975141811986"
## [1] "Iteration: 4"
## [1] "mean: 1.96472386868012"
## [1] "Std Dev: 2.96518430880214"
## [1] "Iteration: 5"
## [1] "mean: 1.98143301600266"
## [1] "Std Dev: 2.87573234590543"
## [1] "Iteration: 6"
## [1] "mean: 2.00828097868113"
## [1] "Std Dev: 3.02648411476097"
## [1] "Iteration: 7"
## [1] "mean: 2.0442036923372"
## [1] "Std Dev: 2.95191979376036"
## [1] "Iteration: 8"
## [1] "mean: 1.94493969423164"
## [1] "Std Dev: 2.91145378312645"
## [1] "Iteration: 9"
## [1] "mean: 2.02687931113431"
## [1] "Std Dev: 2.95608869841762"
## [1] "Iteration: 10"
## [1] "mean: 2.01943668875613"
## [1] "Std Dev: 3.04271640151637"
\end{verbatim}

\section{Question 2}\label{question-2}

\href{https://en.wikipedia.org/wiki/Francis_Galton}{Sir Francis Galton}
was a controversial genius who discovered the phenomenon of ``Regression
to the Mean.'' In this problem, we will examine some of the data that
illustrates the principle.

\subsection{Question 2.1}\label{question-2.1}

First, install and load the library \texttt{HistData} that contains many
famous historical data sets. Then load the Galton data using the command
\texttt{data(Galton)}. Take a look at the first few rows of
\texttt{Galton} data using the command \texttt{head(Galton)}.

\subsection{\texorpdfstring{\textcolor{red}{Solution:}}{}}\label{section-4}

\begin{Shaded}
\begin{Highlighting}[]
\KeywordTok{library}\NormalTok{(}\StringTok{"HistData"}\NormalTok{)}
\KeywordTok{data}\NormalTok{(}\StringTok{"Galton"}\NormalTok{)}
\KeywordTok{head}\NormalTok{(Galton)}
\end{Highlighting}
\end{Shaded}

\begin{verbatim}
##   parent child
## 1   70.5  61.7
## 2   68.5  61.7
## 3   65.5  61.7
## 4   64.5  61.7
## 5   64.0  61.7
## 6   67.5  62.2
\end{verbatim}

As you can see, the data consist of two columns. One is the height of a
parent, and the second is the height of a child. Both heights are
measured in inches.

Plot one histogram of the heights of the children and one histogram of
the heights of the parents. These histograms should use the same
\texttt{x} and \texttt{y} scales.

\subsection{\texorpdfstring{\textcolor{red}{Solution:}}{}}\label{section-5}

\begin{Shaded}
\begin{Highlighting}[]
\KeywordTok{hist}\NormalTok{(Galton}\OperatorTok{$}\NormalTok{parent, }\DataTypeTok{freq =} \OtherTok{FALSE}\NormalTok{)}
\end{Highlighting}
\end{Shaded}

\includegraphics{A1_files/figure-latex/unnamed-chunk-6-1.pdf}

\begin{Shaded}
\begin{Highlighting}[]
\KeywordTok{hist}\NormalTok{(Galton}\OperatorTok{$}\NormalTok{child, }\DataTypeTok{freq =} \OtherTok{FALSE}\NormalTok{)}
\end{Highlighting}
\end{Shaded}

\includegraphics{A1_files/figure-latex/unnamed-chunk-6-2.pdf}

Comment on the shapes of the histograms.

\subsection{\texorpdfstring{\textcolor{red}{Solution:}}{}}\label{section-6}

The parent histogram is strongly centered around 67 to 70 inches. In
all, the data ranges from 64 to 73 inches.

Conversely, there is greater variation in the distribution of children's
heights. The data spans from 61 to 74 inches with a concentration
between 66 and 71 inches. This histogram notably has a flatter
distribution relative to the parent's height distribution.

\subsection{Question 2.2}\label{question-2.2}

Make a scatterplot the height of the child as a function of the height
of the parent. Label the \texttt{x}-axis ``Parent Height (inches),'' and
label the \texttt{y}-axis ``Child Height (inches).'' Give the plot a
main tile of ``Galton Data.''

\subsection{\texorpdfstring{\textcolor{red}{Solution:}}{}}\label{section-7}

\begin{Shaded}
\begin{Highlighting}[]
\KeywordTok{plot}\NormalTok{(}\DataTypeTok{y =}\NormalTok{ Galton}\OperatorTok{$}\NormalTok{child, }\DataTypeTok{x =}\NormalTok{ Galton}\OperatorTok{$}\NormalTok{parent, }\DataTypeTok{xlab =} \StringTok{"Parent Height (inches)"}\NormalTok{, }\DataTypeTok{ylab =} \StringTok{"Child Height (inches)"}\NormalTok{, }\DataTypeTok{main =} \StringTok{"Galton Data"}\NormalTok{)}
\end{Highlighting}
\end{Shaded}

\includegraphics{A1_files/figure-latex/unnamed-chunk-7-1.pdf}

\section{Question 3}\label{question-3}

If necessary, install the \texttt{ISwR} package, and then
\texttt{attach} the \texttt{bp.obese} data from the package. The data
frame has 102 rows and 3 columns. It contains data from a random sample
of Mexican-American adults in a small California town.

\subsection{Question 3.1}\label{question-3.1}

The variable \texttt{sex} is an integer code with 0 representing male
and 1 representing female. Use the \texttt{table} function operation on
the variable `sex' to display how many men and women are represented in
the sample.

\subsection{\texorpdfstring{\textcolor{red}{Solution:}}{}}\label{section-8}

\begin{Shaded}
\begin{Highlighting}[]
\KeywordTok{library}\NormalTok{(}\StringTok{"ISwR"}\NormalTok{)}
\KeywordTok{data}\NormalTok{(}\StringTok{"bp.obese"}\NormalTok{)}
\KeywordTok{table}\NormalTok{(bp.obese}\OperatorTok{$}\NormalTok{sex)}
\end{Highlighting}
\end{Shaded}

\begin{verbatim}
## 
##  0  1 
## 44 58
\end{verbatim}

\subsection{Question 3.2}\label{question-3.2}

The \texttt{cut} function can convert a continuous variable into a
categorical one. Convert the blood pressure variable \texttt{bp} into a
categorical variable called \texttt{bpc} with break points at 80, 120,
and 240. Rename the levels of \texttt{bpc} using the command
\texttt{levels(bpc)\ \textless{}-\ c("low",\ "high")}.

\subsection{\texorpdfstring{\textcolor{red}{Solution:}}{}}\label{section-9}

\begin{Shaded}
\begin{Highlighting}[]
\NormalTok{bpc <-}\StringTok{ }\KeywordTok{cut}\NormalTok{(bp.obese}\OperatorTok{$}\NormalTok{bp, }\DataTypeTok{breaks =} \KeywordTok{c}\NormalTok{(}\DecValTok{80}\NormalTok{, }\DecValTok{120}\NormalTok{, }\DecValTok{240}\NormalTok{))}
\KeywordTok{levels}\NormalTok{(bpc) <-}\StringTok{ }\KeywordTok{c}\NormalTok{(}\StringTok{"low"}\NormalTok{, }\StringTok{"high"}\NormalTok{)}
\end{Highlighting}
\end{Shaded}

\subsection{Question 3.3}\label{question-3.3}

Use the \texttt{table} function to display a relationship between
\texttt{sex} and \texttt{bpc}.

\subsection{\texorpdfstring{\textcolor{red}{Solution:}}{}}\label{section-10}

\begin{Shaded}
\begin{Highlighting}[]
\KeywordTok{table}\NormalTok{(bp.obese}\OperatorTok{$}\NormalTok{sex, bpc)}
\end{Highlighting}
\end{Shaded}

\begin{verbatim}
##    bpc
##     low high
##   0  16   28
##   1  28   30
\end{verbatim}

\subsection{Question 3.4}\label{question-3.4}

Now cut the \texttt{obese} variable into a categorical variable
\texttt{obesec} with break points 0, 1.25, and 2.5. Rename the levels of
\texttt{obesec} using the command
\texttt{levels(obesec)\ \textless{}-\ c("low",\ "high")}.

Use the \texttt{ftable} function to display a 3-way relationship between
\texttt{sex}, \texttt{bpc}, and \texttt{obesec}.

\subsection{\texorpdfstring{\textcolor{red}{Solution:}}{}}\label{section-11}

\begin{Shaded}
\begin{Highlighting}[]
\NormalTok{obesec <-}\StringTok{ }\KeywordTok{cut}\NormalTok{(bp.obese}\OperatorTok{$}\NormalTok{obese, }\DataTypeTok{breaks =} \KeywordTok{c}\NormalTok{(}\DecValTok{0}\NormalTok{, }\FloatTok{1.25}\NormalTok{, }\FloatTok{2.5}\NormalTok{))}
\KeywordTok{levels}\NormalTok{(obesec) <-}\StringTok{ }\KeywordTok{c}\NormalTok{(}\StringTok{"low"}\NormalTok{, }\StringTok{"high"}\NormalTok{)}

\KeywordTok{ftable}\NormalTok{(bp.obese}\OperatorTok{$}\NormalTok{sex, bpc, obesec)}
\end{Highlighting}
\end{Shaded}

\begin{verbatim}
##        obesec low high
##   bpc                 
## 0 low          12    4
##   high         15   13
## 1 low          14   14
##   high          4   26
\end{verbatim}

Which group do you think is most at risk of suffering from obesity?

\subsection{\texorpdfstring{\textcolor{red}{Solution:}}{}}\label{section-12}

Proportions of Obese Men and Women in Sample:

\begin{Shaded}
\begin{Highlighting}[]
\KeywordTok{table}\NormalTok{(bp.obese}\OperatorTok{$}\NormalTok{sex, obesec)}
\end{Highlighting}
\end{Shaded}

\begin{verbatim}
##    obesec
##     low high
##   0  27   17
##   1  18   40
\end{verbatim}

\begin{Shaded}
\begin{Highlighting}[]
\DecValTok{17} \OperatorTok{/}\StringTok{ }\NormalTok{(}\DecValTok{17} \OperatorTok{+}\StringTok{ }\DecValTok{27}\NormalTok{) }\CommentTok{# % of Obese Men}
\end{Highlighting}
\end{Shaded}

\begin{verbatim}
## [1] 0.3863636
\end{verbatim}

\begin{Shaded}
\begin{Highlighting}[]
\DecValTok{40} \OperatorTok{/}\StringTok{ }\NormalTok{(}\DecValTok{40} \OperatorTok{+}\StringTok{ }\DecValTok{18}\NormalTok{) }\CommentTok{# % of Obese Men}
\end{Highlighting}
\end{Shaded}

\begin{verbatim}
## [1] 0.6896552
\end{verbatim}

Regardless of blood pressure, women are more likely to be obese than men
(based on the sample data proportions: 68\% vs 38\%).

Proportions of Obese High and Low blood pressure individuals in Sample:

\begin{Shaded}
\begin{Highlighting}[]
\KeywordTok{table}\NormalTok{(bpc, obesec)}
\end{Highlighting}
\end{Shaded}

\begin{verbatim}
##       obesec
## bpc    low high
##   low   26   18
##   high  19   39
\end{verbatim}

\begin{Shaded}
\begin{Highlighting}[]
\DecValTok{18} \OperatorTok{/}\StringTok{ }\NormalTok{(}\DecValTok{18} \OperatorTok{+}\StringTok{ }\DecValTok{26}\NormalTok{)  }\CommentTok{# % of Obese Individuals with Low blood pressure}
\end{Highlighting}
\end{Shaded}

\begin{verbatim}
## [1] 0.4090909
\end{verbatim}

\begin{Shaded}
\begin{Highlighting}[]
\DecValTok{39} \OperatorTok{/}\StringTok{ }\NormalTok{(}\DecValTok{39} \OperatorTok{+}\StringTok{ }\DecValTok{19}\NormalTok{)  }\CommentTok{# % of Obese Individuals with High blood pressure}
\end{Highlighting}
\end{Shaded}

\begin{verbatim}
## [1] 0.6724138
\end{verbatim}

Regardless of sex, individuals with high blood pressure are more likely
to be obese than individuals with low blood pressure (based on the
sample data proportions: 67\% vs 41\%).

Proportions of Obese Men/Women by blood pressure in Sample:

\begin{Shaded}
\begin{Highlighting}[]
\KeywordTok{ftable}\NormalTok{(bp.obese}\OperatorTok{$}\NormalTok{sex, bpc, obesec)}
\end{Highlighting}
\end{Shaded}

\begin{verbatim}
##        obesec low high
##   bpc                 
## 0 low          12    4
##   high         15   13
## 1 low          14   14
##   high          4   26
\end{verbatim}

\begin{Shaded}
\begin{Highlighting}[]
\CommentTok{# Proportion of Obese Men}
\DecValTok{4} \OperatorTok{/}\StringTok{ }\NormalTok{(}\DecValTok{4} \OperatorTok{+}\StringTok{ }\DecValTok{12}\NormalTok{)   }\CommentTok{# Men w/ Low Blood Pressure}
\end{Highlighting}
\end{Shaded}

\begin{verbatim}
## [1] 0.25
\end{verbatim}

\begin{Shaded}
\begin{Highlighting}[]
\DecValTok{13} \OperatorTok{/}\StringTok{ }\NormalTok{(}\DecValTok{13} \OperatorTok{+}\StringTok{ }\DecValTok{15}\NormalTok{) }\CommentTok{# Men w/ High Blood Pressure}
\end{Highlighting}
\end{Shaded}

\begin{verbatim}
## [1] 0.4642857
\end{verbatim}

\begin{Shaded}
\begin{Highlighting}[]
\CommentTok{# Proportion of Obese Women}
\DecValTok{14} \OperatorTok{/}\StringTok{ }\NormalTok{(}\DecValTok{14} \OperatorTok{+}\StringTok{ }\DecValTok{14}\NormalTok{)}\CommentTok{# Women w/ Low Blood Pressure}
\end{Highlighting}
\end{Shaded}

\begin{verbatim}
## [1] 0.5
\end{verbatim}

\begin{Shaded}
\begin{Highlighting}[]
\DecValTok{26} \OperatorTok{/}\StringTok{ }\NormalTok{(}\DecValTok{26} \OperatorTok{+}\StringTok{ }\DecValTok{4}\NormalTok{) }\CommentTok{# Women w/ High Blood Pressure}
\end{Highlighting}
\end{Shaded}

\begin{verbatim}
## [1] 0.8666667
\end{verbatim}

For Men and Women of comparable Blood Pressure classifications, the
sample proportion of obese women is always higher (for both high and low
blood pressure categories).

Given that the sample was taken from a small town of Mexican-American
adults in southern California, these results most closely apply to
individuals of this geographical location and ethnic background. With
this context in mind, the female population (regardless of blood
pressure category) is most at risk of suffering from obesity because of
the comparative proportions analyzed above.


\end{document}
